\documentclass[11pt, a4paper]{article}
\usepackage[slovene]{babel}
\usepackage[T1]{fontenc}
\usepackage[utf8]{inputenc}
\usepackage{nopageno}
\usepackage{lmodern}
\usepackage{eurosym}
\usepackage{amssymb}
\usepackage{amsfonts}
\usepackage{amsmath}
\usepackage{graphicx}
\DeclareUnicodeCharacter{2212}{-}
\usepackage{graphicx}
\usepackage{subfig}

\begin{document}

\begin{titlepage}
\begin{center}

\Huge 
\textbf{Skupina 22: \textsl{k}-means algoritem}

\vspace{1cm}
\Large
\textbf{Opis problema in načrt dela}

\vspace{1cm}
\large
Projekt v povezavi s predmetom Operacijske raziskave


\vspace{2,5cm}
\large
Avtorja:\\
\textbf{Matevž Raspet, Eva Šraj}\\

\vfill

\Large Ljubljana, november 2018

\end{center}
\end{titlepage}

\newpage

\section{\textbf{OPIS PROBLEMA in NAČRT DELA}}

\noindent Pri projektu iz predmeta Operacijskih raziskav bova preučevala delovanje \textsl{k}-means algoritma. \\ 

\noindent \textsl{K}-means algoritem je metoda vektorske kvantizacije, namenjena za analizo grupiranja podatkov, predvsem pri rudarjenju s podatki. Ta algoritem naredi particijo $n$ meritev v $k$ različnih grup, v katerih vsaka meritev pripada grupi z najbližjo povprečno vrednostjo. Rezultat particije je razdelitev prostora v Voronojeve celice. \\
\noindent Bolj formalno: naj bodo $(x_1, x_2, \ldots,x_n)$ dane meritve, kjer je vsaka meritev $x_i$ $d$-dimenzionalen realni vektor. \textsl{K}-means algoritem napravi particijo $n$ meritev v $k  (\leq n)$ množic $S= \{S_1, S_2, \ldots, S_k\}$, na podlagi najmanjše vsote kvadratov razdalj med meritvami posamezne množice in njihovim povprečjem $\mu_i$.

\begin{align*}
\arg\min_{S} \sum_{i = 1}^k \sum_{x \in S_i} \| x - \mu_i \|^2 = \arg \min_{S} \sum_{i = 1}^k |S_i| \cdot Var(S_i),
\end{align*}


\noindent S pomočjo dveh metod bova zgenerirala in predstavila naključne podatke v prostorih $\mathbb{R}$, $\mathbb{R}^2$ in $\mathbb{R}^3$. Ti dve metodi sta: \textbf{metoda Manhattan} in \textbf{metoda najmanjših kvadratov} (krajše MNK). Uporabila ju bova na praktičnih življenjskih primerih, kot npr. \textsl{problem najbolj obiskane trgovine}. \\

\noindent Nato bova \textsl{k}-means algoritem uporabila na množicah, z 10\% najbolj odstopajočimi meritami ter primerjala rezultata dobljena z Manhattan metodo in MNK. \\

\noindent Na koncu bova za množice, ki so generirane v prostoru $\mathbb{R}^2$, poiskala povezavo med \textsl{k}-means algoritmom in \textsc{Voronojevim diagramom} ter rešitev prikazala grafično.

\subsection{\textbf{NEZNANI POJMI}}

\noindent Kot sva omenila že zgoraj, bova pri projektu uporabila 2 metodi za računanje razdalj:
\begin{itemize}
	\item Manhattan metodo in
	\item metodo najmanjših kvadratov.
\end{itemize}

\newpage

\noindent \textbf{Definicija:} Naj bosta $X$ in $Y$ $n$-razsežna vektorja; $X = (x_1, x_2, \ldots, x_n)$ in $Y = (y_1, y_2,\ldots, y_n).$ \\
\noindent Razdalja po \textbf{metodi Manhattan} je definirana kot: 
\begin{align*}
d(X, Y) = \sum_{i = 1}^n | x_i - y_i |
\end{align*}
\noindent Po \textbf{metodi najmanjših kvadratov} razdaljo definiramo kot:
\begin{align*}
d(X, Y) = \sqrt {\sum_{i = 1}^n (x_i - y_i)^2}
\end{align*}

\noindent \textsc{Voronojev diagram}

\noindent V matematiki Voronojev diagram prikazuje razdelitev ploskve na območja, osnovana na razdalji točk v posameznih podmnožicah območij. 

\vspace{0,5 cm}

\noindent \textbf{Definicija:} Naj bo $X$ metrični prostor z razdaljo $d$, $K$ indeksna množica in $P_k$ $(k \in K)$ nabor nepraznih podmnožic v (metričnem) prostoru $X$. Voronoijeva celica ali območje $R_k$ povezano s $P_k$, je množica točk v $X$, katerih razdalja do $P_k$ ni večja od razdalj do drugih $P_j$, kjer je $j$ poljuben indeks različen od k $(j \neq k)$. \\

\noindent Z drugimi besedami: če je $d(x,A) = \inf\{d(x,a) | a \in A\}$ razdalja med $X$ in $A$, je nato Voronojev diagram nabor celic $(R_k)_{k \in K}$, kjer je

\begin{equation*}
R_k = \{ x \in X | d(x, P_k) \leq d(x, P_j)   za vse j \neq k \}
\end{equation*}

\subsection{\textbf{POTEK DELA S PRAKTIČNIM PRIMEROM}}

\noindent Kot enostaven primer bova preučevala \textsl{problem najbolj obiskane trgovine}. \\
\noindent Za množico bova vzela trgovine v nekem mestu. Predpostavila bova, da imajo v vseh trgovinah produkte z enako ceno in kvaliteto. Razumljivo je, da je kupčeva izbira odvisna le od njegove razdalje do trgovin (kupci bodo kupovali v njim najbližji trgovini). Midva pa želiva oceniti število kupcev v posamezni trgovini.

\vspace{0,5 cm}

\noindent V tem primeru je Voronoijeva celica/območje $R_k$ dane trgovine $P_k$ približna ocena števila kupcev, ki bodo kupovali v tej trgovini $ P_k$ (trgovine so v našem modelu predstavljene kot točke v mestu). \\
\noindent Za večino mest bova razdaljo med točkami izmerila s pomočjo metode Manhattan in MNK.
Pripadajoča grafa izgledata približno takole:

\begin{figure}[tbp]
\centering
\includegraphics[width = 7cm]{Euclidean_Voronoi_Diagram.png}
\caption{Voronojev diagram z MNK}
\end{figure}

\begin{figure}[tbp]
\centering
\includegraphics[width = 7cm]{Manhattan_Voronoi_Diagram.png}
\caption{Voronojev diagram z Manhattan metodo}
\end{figure}


\end{document}